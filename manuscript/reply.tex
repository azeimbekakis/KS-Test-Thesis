\documentclass[12pt]{article}
\usepackage[margin=1in]{geometry}
\usepackage{natbib}
\usepackage{hyperref}

\usepackage{listings}
\usepackage{rotating, graphicx}
\graphicspath{{./}, {./image/}}
\usepackage{booktabs, natbib}
% \usepackage{breakurl}
% \usepackage [english]{babel}
\usepackage{amsmath, amsbsy, amsthm, epsfig, epsf, psfrag, graphicx, 
amssymb, enumerate}
\usepackage{bm}
\usepackage{multirow, multicol}

\usepackage[dvipsnames]{color}
\definecolor{darkblue}{rgb}{0.1, 0.2, 0.6}

\newcommand{\jy}[1]{\textcolor{red}{JY: #1}}
\newcommand{\eds}[1]{\textcolor{blue}{(EDS: #1)}}
\newcommand{\mc}[1]{\textcolor{green}{(MC: #1)}}

\sloppy

% \usepackage{csquotes}
% \usepackage [autostyle, english = american]{csquotes}
% \MakeOuterQuote{"}

% \usepackage{bibentry}
\newenvironment{comment}%
{\begin{quotation}\noindent\small\it\color{darkblue}\ignorespaces%
}{\end{quotation}}


\begin{document}

\begin{center}
  {\Large\bf Response to the Comments}
\end{center}

We extend our gratitude to the Editor and Associate Editor for 
granting us the opportunity to revise this manuscript. We also want to
express our appreciation to the reviewer for their valuable comments. 


The manuscript has been revised accordingly with the most notable change being 
the addition of Section 5 (Classroom Implementation). We believe this revision 
has enhanced the both the quality and utility of this paper.


Point-by-point responses to the comments are as follows, with the
comments quoted in \emph{\color{darkblue} italic and blue}.

\subsection*{To Associate Editor}

\begin{comment}
In this article, the author explains how the Kolmogorov-Smirnov (KS) test is 
often misused in practice and show through simulation, that under the null 
hypothesis the KS p-value is very likely to not have a uniform distribution if 
the parameters are estimated from the data or when the data are serial 
correlated. The authors also give solutions to correct the test in both of these 
situations using the parametric or semiparametric bootstrap.


I found the article informative and well written as the problem and proposed 
solutions were clearly explained and demonstrated. It seemed to me there is 
possibly enough there to warrant publishing the article. However, as it was 
submitted to Teacher’s corner, I wanted an opinion on the value of the article
as a teaching tool.


Therefore, the article was reviewed by one referee who is an educator and could 
assess the impact and usefulness on teaching (I tried for two, but that proved 
hard to come by). This referee also found the article clear and well written, 
but wanted the authors to include a detailed explanation of how to present
this material to students and where/how to include it in the course material.


It would seem that this could be taught while teaching statistical tests and 
methods as a clear warning to students to always assess how well their data fits 
the assumptions of the method. But how to present this? It sounds like, from the 
referee’s opinion, that this would require significant work. One way to do this 
is if the author included a new section dedicated to the implementing of this 
method in the classroom. As the rest of the article seems clear and informative, 
the rest of paper could be left mostly unchanged.
\end{comment}


Thank you very much for your encouraging comments and great suggestion.  
We have now included a new section, \textit{Section 5 Classroom Implementation},
that discusses how elements of this manuscript have been incorporated into both 
graduate- and undergraduate-level teaching at the authors' home university. 
 

For blinding purposes, course numbers, author initials and websites, and 
university name have been redacted, but all this information appears in the 
unblinded version.  In brief, this material has been covered at the graduate
level in a statistical computing course, open to all graduate students in 
Statistics and others with permission, that covers the bootstrap. Detailed R 
code and graphical illustrations were provided to students in the open-access 
course notes to elucidate the distinctions between employing null
hypotheses with and without defined parameters.  This link is included in the 
unblinded version of the paper. \eds{Not sure how we can blind your github 
website for the submission unless we actually don't include it in the blinded 
version. Could we print to PDF the relevant section of the notes and include as 
a supplement for review purposes?} 


Similarly at the undergraduate-level, students in an R-based applied regression 
course were provided with an R Markdown file with simulation code in order to 
demonstrate how the KS test fails with estimated parameters in the hypothesized 
distribution. The R Markdown file that was provided to students is included in 
the Supplemental Materials. This R-based applied linear regression course are 
has the following prerequisites: two semesters of introductory statistics OR a 
Calculus~2-based statistical methods course.


At both the graduate and undergraduate level, students were able to follow the 
demonstrations closely and gain a clear understanding of the implications of 
using estimated parameters within test statistics.  At the graduate level, 
students further were able to appreciate the efficacy of the parametric 
bootstrap method as a corrective measure.



\subsection*{To Reviewer}


\begin{comment}
In this paper we read about a misfit between the KS test as it was developed and 
how it is often used in practice. The author shows, via simulation, that the KS 
p-value does not have a uniform distribution (as it should) when H0 is true but 
parameters are estimated from the data, nor when there is serial correlation 
(or both). At least we see this for two settings: the normal and the gamma. The 
problem is clearly presented, as is the solution (use of the parametric or 
semiparametric bootstrap).
\end{comment}


Thank you for this encouraging comment.  


\begin{comment}
The figures are easy to read and convey the necessary information. (It would 
help if Table 1 appeared on page 14, rather than page 13.)
\end{comment}


This is a very good point.  We have moved the location of Table 1 to be closer 
to the end of Section 4 (page 15 in the revised manuscript).  


\begin{comment}
Since this is a Teacher's Corner submission, I was hoping to read about how 
these ideas have been presented in specific courses and also how students 
reacted to this. What background did the students have? What level of detail 
was given to them? That is, it is one thing to say "use the bootstrap" and 
another thing to present to students, e.g., a Markdown file with all necessary
code included. Was that done? Were students invited to conduct their own 
simulations? Etc.
\end{comment} 


Thank you for highlighting this important omission. Taking the suggestion of the 
Associate Editor, we have now included a new section, Section 5 Classroom 
Implementation, that discusses how elements of
this manuscript have been incorporated into both graduate- and 
undergraduate-level teaching at the authors' home university. 
Please see our response to the Associate Editor for additional details. 

%\bibliographystyle{chicago}
%\bibliography{citations}


\end{document}
